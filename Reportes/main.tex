\documentclass{article}
\usepackage[utf8]{inputenc}
\usepackage[spanish]{babel}
\usepackage{listings}

\begin{document}

\begin{titlepage}
    \begin{center}
        \vspace*{1cm}
            
        \Huge
        \textbf{Informe de análisis y diseño}
            
        \vspace{0.5cm}
        \LARGE
        Parcial 2
            
        \vspace{1.5cm}
            
        \textbf{Angie Paola Jaramillo Ortega}
            
        \vfill
            
        \vspace{0.8cm}
            
        \Large
        Despartamento de Ingeniería Electrónica y Telecomunicaciones\\
        Universidad de Antioquia\\
        Medellín\\
        Septiembre de 2021
            
    \end{center}
\end{titlepage}

\tableofcontents
\newpage
\section{Sección introductoria}\label{intro}
Se ha asignado el trabajo de crear una pantalla de LED's para la representación de distintas banderas. El programa debe ser capaz de mostrar la bandera escogida por el usuario sin importar su tamaño y debe ser reconocible.

\section{Análisis} \label{Análisis}
Una imagen seleccionada por el usuario debe ser leida por el programa en Qt, se debe procesar la imagen por submuestreo o sobremuestreo para adaptarse a una matriz de LEDs y se debe generar un archivo txt con la porción de código que represente la imagen a mostrar en la pantalla, la información del archivo txt será pasada al codigo de arduino y al iniciar la simulación la bandera debe ser mostrada en la pantalla LED.

\begin{description}
   \item[Datos de entrada:] Nombre de un archivo imagen.jpg 
   \item[Datos de salida en Qt:] Segmento de código a ser agregado en el controlador de la matriz de LEDs de arduino 
   \item[Datos de salida en arduino:] Representación de la imagen ingresada en la matriz de LEDs
\end{description}

Para implementar esta solución se hará uso de la clase QImage para abrir la imagen y leer el su contenido. Se usará los métodos de la clase QImage, width() y heigh() para obtener el ancho y la altura de la imagen respectivamente. El método pixelColor() de la clase QImage servirá para obtener los valores RGB de cada pixel.
Para el código en arduino se utilizará principalmente la función setPixelColor() para especificar el color de cada pixel.

\subsection{Tareas}

\begin{description}
   \item Lectura de imagen
   \item Obtener medidas de la imagen
   \item Definir si es necesario submuestreo o sobremuestreo
   \item Hacer proceso de transformación
   \item Crear archivo txt
   \item En arduino leer la nueva imagen
   \item estblecer valore RGB de cada pixel
   
\end{description}

\subsection{Alternativa de solución}

para el procesamiento de imagenes dependerá de si es necesario realizar submuestreo o sobremuestreo.

\begin{description}
   \item[Submuestreo:] En caso de que se necesite reducir la cantidad de pixeles de la imagen ingresada se realizará promediando los valores RGB de un conjunto de pixeles para ubicarlo en un pixel correspondiente de la matriz de LEDs.
   \item[Sobremuestreo:] En caso de tener que aumentar el tamaño de la bandera se usará la técnica de interpolación bilinial, esta calcula los valores de una ubicación utilizando las cuatro posiciones mas cercanas para realizar un promedio ponderado usando como pesos la distancia a estos cuatro puntos
\end{description}


\subsection{Algoritmo}
%
El siguiente algoritmo \ref{algoritmo}, es un diseño general de la solución planteada. 


\begin{lstlisting}[label=algoritmo]
{
    Escribir "Ingrese nombre de archivo imagen"
    Leer imagen
    Crear objeto imagen
    
    ancho<-leer ancho de imagen
    altura<-leer altura de imagen
    
    SI altura>10 OR ancho>10 entonces
        submuestrear(imagen)
    SINO altura<10 OR ancho<10 entonces
        sobremuestrear(imagen)
    SINO altura=10 AND ancho=10 entonces
        entero array[10][10][3]
        Desde fila=0 hasta 10 con paso 1 hacer
            Desde columna=0 hasta 10 hacer
                ColorR<-leer color rojo de pixel (fila,columna)
                ColorG<-leer color verde de pixel (fila,columna)
                ColorB<-leer color azul de pixel (fila,columna)
                array[fila][columna]<-[colorR,ColorG,ColorB]
    FIN SI

}

\end{lstlisting}

\section{Consideraciones} \label{Consideraciones}
La solución del problema se realizará en una matriz de LEDs 10x10 y la cantidad de pixeles a promediar durante el submuestreo dependerá de la dimensión de la imagen original.Los promedios sacados para el submuestreo como para el sobremuestreo se realizarán para cada componete R,G y B.

\end{document}
